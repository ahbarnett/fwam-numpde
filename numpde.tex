% FWAM lecture. Func Approx & PDE. CCM beamer template.
%
% Build:
% pdflatex -shell-escape numpde; bibtex numpde; pdflatex numpde
%
% (here -shell-escape lets it call epstopdf to convert .eps to .stuff.pdf)

\documentclass[t]{beamer}
\usepackage[utf8]{inputenc}
\usepackage{epstopdf}  % handle EPS

\usepackage{multimedia,hyperref}   % embeds but needs external movie files

\usepackage{bm,tcolorbox}

% Alex's macros for beamer. Use macros to let you type less stuff
\newcommand{\ft}[1]{\frametitle{#1}}
% note: can't make macros for begin/end frame: http://tex.stackexchange.com/questions/326787/difficulty-in-creating-macro-newcommand-for-the-beginning-and-end-of-frames-i

% frame box on next slide
\newcommand<>{\tmpbox}[1]{
    \definecolor{colorbox}{RGB}{255,255,255}
    \only#2{\definecolor{colorbox}{RGB}{255,128,0}}
    \tcbox[left=0pt,right=0pt,top=0pt,bottom=0pt,colback=white,colframe=colorbox,nobeforeafter,tcbox raise base]{#1}
}
% https://tex.stackexchange.com/questions/186044/how-a-framebox-will-appear-around-the-text-on-a-click

\input{rgb}
\newcommand{\lb}[1]{{\color{blue}#1}}    % from prosper azure titles
\newcommand{\bi}{\begin{itemize}}
\newcommand{\ei}{\end{itemize}}
\newcommand{\ben}{\begin{enumerate}}
\newcommand{\een}{\end{enumerate}}
\newcommand{\be}{\begin{equation}}
\newcommand{\ee}{\end{equation}}
\newcommand{\bea}{\begin{eqnarray}}
\newcommand{\eea}{\end{eqnarray}}
\newcommand{\bc}{\begin{center}}
\newcommand{\ec}{\end{center}}
\newcommand{\mbf}[1]{{\bm #1}}           % requires bm package
\newtheorem{thm}{Theorem}
\newcommand{\ig}[2]{\includegraphics[#1]{#2}}
\newcommand{\tbox}[1]{{\mbox{\tiny #1}}}
\newcommand{\cbox}[1]{{\mbox{\scriptsize #1}}}
%\newcommand{\who}[1]{{\tiny \textcolor{green}{(#1)}}}
\newcommand{\who}[1]{{\scriptsize \textcolor{darkgreen}{(#1)}}}
\newcommand{\whoc}[1]{{\scriptsize \textcolor{darkgreen}{#1}}} % use w/ \cite
\newcommand{\com}[1]{{\scriptsize \textcolor{purple}{#1}}}      % comment
\newcommand{\co}[1]{\mbox{\scriptsize \tt \textcolor{black} #1}}          % code
\newcommand{\vg}{\vspace{2ex}}
\newcommand{\sg}{\vspace{1ex}}
\newcommand{\hg}{\vspace{0.5ex}}
\newcommand{\gb}{\ensuremath{\textcolor{darkgreen}{\bullet\;}}\ }
\newcommand{\bmp}[1]{\begin{minipage}{#1}}
\newcommand{\bmpt}[1]{\begin{minipage}[t]{#1}}
\newcommand{\emp}{\end{minipage}}
\newcommand{\pig}[2]{\bmp{#1}\includegraphics[width=#1]{#2}\emp} % mp-fig, nogap
\newcommand{\pigm}[3]{\bmp{#1}\href{#3}{\includegraphics[width=#1]{#2}}\emp} % w/ movie
\newcommand{\ora}[1]{{\color{CCMorange} #1}}
\newcommand{\gre}[1]{{\color{darkgreen} #1}}
\newcommand{\yel}[1]{{\color{yellow} #1}}
\newcommand{\red}[1]{{\color{red} #1}}
\newcommand{\sr}[1]{{\scriptsize #1}}
\newcommand{\vt}[2]{\biggl[\begin{matrix}#1\\#2\end{matrix}\biggr]} % 2-col-vec
\newcommand{\mt}[4]{\biggl[\begin{matrix}#1&#2\\#3&#4\end{matrix}\biggr]} %2x2

% macros for this work
\newcommand{\RR}{\mathbb{R}}
\newcommand{\sfrac}[2]{\mbox{\small $\frac{#1}{#2}$}}
\newcommand{\half}{\sfrac{1}{2}}
\newcommand{\bigO}{{\mathcal O}}
\newcommand{\eps}{\varepsilon}
\newcommand{\x}{\mbf{x}}
\newcommand{\y}{\mbf{y}}


\DeclareMathOperator{\im}{Im}
\DeclareMathOperator{\re}{Re}

\title{FWAM Session B: Function Approximation and Differential Equations}
\date{Wednesday afternoon, 10/30/19}
\author{\textbf{Alex Barnett}\inst{1} and \textbf{Keaton Burns}\inst{2}}
\institute{\inst{1} Center for Computational Mathematics, Flatiron Institute\\
  \inst{2} Center for Computational Astrophysics, Flatiron Institute,
  and Department of Mathematics, MIT
}

\usetheme{CCM}

% see: https://tex.stackexchange.com/questions/124256/how-do-i-get-numbered-entries-in-a-beamer-bibliography
\setbeamertemplate{bibliography item}{{\textcolor{darkgreen}\insertbiblabel}}

\begin{document}

\begin{frame}
  \titlepage
\end{frame}


%Abstract:
%I overview key concepts and practical methods for efficient and accurate numerical function approximation, integration and differentiation. This is the basis for spectral and other ODE/PDE solvers coming up in the next talk. I will teach concepts such as convergence rate, local/global, adaptivity, rounding error, polynomial and Fourier bases. The focus is on 1D, with pointers to higher-dimensional methods and codes.


% ---------------------------------------------------
\begin{frame}\ft{}

\vspace{10ex}

\bc
\ora{  LECTURE 1
\vg

interpolation, integration, differentiation, spectral methods
}
\ec

\end{frame}

% ---------------------------------------------------
\begin{noframe}\ft{Goals and plan}
  
  \lb{Overall}: graph of $f(x)$ needs $\infty$ number of points to describe, so how

  handle $f$ to user-specified accuracy in computer w/ least cost? \com{(bytes/flops)}
  
\pause
  
  \bi

\item
  Interpolation:  also key to numerical ODE/PDEs\ldots

task: given exact $f(x_j)$ at some $x_j$, model $f(x)$ at other points $x$?

\quad\com{App: cheap but accurate ``look-up table'' for possibly expensive func.}

\quad\com{Contrast: fit noisy data = 
  learning (pdf for) params in model, via likelihood/prior}

\item Numerical integration:

  \quad\com{App: computing expectation values, given a pdf or quantum wavefunc.}

  \quad\com{App: integral equation methods for PDEs \;\who{Jun Wang's talk}}
  
\item Numerical differentiation:

  \quad\com{App: build a matrix (linear system) to approximate an ODE/PDE\; \who{Lecture II}}

  \quad\com{App: get gradient $\nabla f$, eg for optimization \;
    \who{cf adjoint methods}
  }
\ei  

\pause
\sg

\lb{Key concepts:} %key throughout numerical analysis}:

convergence rate, degree of smoothness of $f$, global vs local,

spectral methods, adaptivity, rounding error \& catastrophic cancellation

\sg
\pause

\lb{Plus:} good 1D tools, pointers to codes, higher dim methods, opinions!

\end{noframe}


% ---------------------------------------------------
\begin{noframe}\ft{Interpolation in 1D ($d=1$)}

\bmp{3in}
  Say $y_j = f(x_j)$ known at nodes $\{x_j\}$ \quad \com{$N$-pt ``grid''}

  \quad\com{note: exact data, not noisy}
  
  want interpolant $\tilde f(x)$, s.t.\ $\tilde f(x_j)=y_j$
  \emp
  \hfill
\pig{1.5in}{figs/fsamp}

\pause
\sg

\bmp{3.9in}
  hopeless w/o assumptions on $f$, eg smoothness, otherwise\dots

  \gb extra info helps, eg $f$ periodic, or $f(x) = \mbox{smooth}\cdot|x|^{-1/2}$
  \emp
  \hfill
\pig{.6in}{figs/fcrazy}

\pause
\vg

\bmp{3in}
\lb{Simplest}: use value at $x_j$ nearest to $x$

\hfill
\com{``snap to grid''}
%\com{$h=x_{j+1}-x_j$}

Error $\max_x|\tilde f(x)-f(x)| = \bigO(h)$ as $h\to 0$

\quad \com{holds if $f'$ bounded; ie $f$ can be nonsmooth but not crazy}
  
\emp
\hfill
\pig{1.3in}{figs/fsnap}

\sg

\gre{Recap %asymptotic
notation
  ``$\bigO(h)$'': exists $C,h_0$ s.t.\ error $\le Ch$ for all $0<h<h_0$}

\pause
\vg

\bmp{3in}
\lb{Piecewise linear}:
\hfill \com{``connect the dots''}

\sg

max error $ = \bigO(h^2)$ as $h\to 0$

\quad \com{needs $f''$ bounded, ie smoother than before}
  
\emp
\hfill
\pig{1.3in}{figs/flin}

\sg

Message: a higher order method is {\em only} higher order if $f$ smooth enough

\end{noframe}



% ---------------------------------------------------
\begin{noframe}\ft{Interlude: convergence rates}

  Should know or measure convergence rate of any method you use

\gb ``effort'' parameter $N$ \hfill \com{eg \# grid-points = $1/h^d$ where $h = $ grid spacing,  $d=$ dim}

We just saw algebraic conv.\ error $=\bigO(N^{-p})$, for order $p=1,2$

\pause

There's only one graph in numerical analysis: \framebox{``relative error vs effort''}

\sg

\pig{2.2in}{figs/convloglog}
\pause
\hfill
\pig{2.2in}{figs/convlinlog}
\pause

\sg

Note how spectral gets many digits for small $N$
\hfill \com{crucial for eg 3D prob.}
%Let's say a 3D prob.\ must have $N\le 300$: here spectral is 

\com{``spectral'' = ``superalgebraic'', beats $\bigO(N^{-p})$ for any $p$}

\gb how many digits to you want? \com{for 1-digit (10\% error), low order ok, easier to code}

\vg
\pause

\mbox{{\tt \red{<rant>}}
test your code w/ {\em known exact soln} to check error conv.
{\tt \red{<\textbackslash rant>}} }

\quad \com{How big is prefactor $C$ in error $\le Ch^p$ ? \; Has asymp.\ rate even kicked in yet? :)}

%You may not believe your model, but at least should trust its soln

\end{noframe}

% ---------------------------------------------------
\begin{frame}\ft{Higher-order interpolation for smooth $f$: the local idea}

  Pick a $p$, eg 6. \; For any target $x$, use only the nearest $p$ nodes:

  \bmp{2.9in}

  Exists unique degree-$(p{-}1)$ poly, $\sum_{k=0}^{p-1} c_k x^k$
  
  \quad which matches local data $(x_j,y_j)_{j=1}^p$
  
  \quad \com{generalizes piecewise lin. idea}

  \quad \com{do \red{not} eval poly outside its central region!}
  
  \emp
  \hfill
  \pig{1.6in}{figs/flocpoly}
  \pause
  
  \gb error $\bigO(h^{p})$, ie high order, but $\tilde f$ {\em not} continuous ($\tilde f \notin C$) \hfill\com{has small jumps}

  \quad \com{if must have cont, recommend splines, eg cubic $p=3$:
    $\tilde f\in C^2$, meaning $\tilde f''$ is cont.}
%   as target $x$ moves, ``jumps'' to another poly,
  
\pause

\vg

\lb{How to find this degree-$(p{-}1)$ poly?}

1) crafty: solve square lin sys for coeffs\quad 
$\sum_{k<p} x_j^k c_k = y_j$ \hfill \com{ $j=1,\dots,p$}

\qquad ie, \; $V\mbf{c} = \y$ \qquad \com{$V$=''Vandermonde'' matrix, is ill-cond.\ but works}

\pause
2) traditional: barycentric formula
$\displaystyle \tilde f(x) = \frac{\sum_{j=1}^p \frac{y_j}{x-x_j}w_j}{\sum_{j=1}^p \frac{1}{x-x_j}w_j}$
\bmp{1in}

\hfill \com{$w_j = \frac{1}{\prod_{i\neq j}(x_j-x_i)}$}

\vg

\hfill\whoc{\cite[Ch.~5]{ATAP}}
\emp

\sg

Either way, $\tilde f(x) = \sum_{j=1}^p y_j \ell_j(x)$
where $\ell_j(x)$ is $j$th Lagrange basis func:

\qquad \pig{2in}{figs/lag}


\end{frame}

% ---------------------------------------------------
\begin{noframe}\ft{Global polynomial (Lagrange) interpolation?}

  Want increase order $p$. Use {\em all} data, get single $\tilde f(x)$, so $p=N$?
  \hfill\com{``global''}

  \pause
  
  $p=N=32$, smooth (analytic) $f(x) = \frac{1}{1+9x^2}$ on $[-1,1]$ : \hfill\who{Runge 1901}
  
  \pig{2.2in}{figs/demopoly1}
  \hfill
  \uncover<3->{
  \pig{2.2in}{figs/demopoly2}
  }
  
\pause
  \gb warning: \red{unif.\ grid, global interp.\ fails} \; \com{$\rightarrow$ only use locally in central region}
%  interval $[a,b]$

  \pause
  \bmp{2.9in}
  But exists good choice of nodes\ldots

\hg
  
  ``Chebychev'': \com{means non-unif.\ grid density $\sim \frac{1}{\sqrt{1-x^2}}$}

  \gb our first spectral method % basis of higher-dim $d>1$ methods\ldots

\quad   max err $=\bigO(\rho^{-N})$
\hfill \com{exponential conv!}
  
\quad \com{  $\rho>1$ ``radius'' of largest
  ellipse in which $f$ analytic}
  
\emp
\hfill
  \pig{1.65in}{figs/bern}

% * give the rate rho = 1.39.. ?  x=1i/3; rho=abs(x + sqrt(x^2-1)), rho^-32
  
% * Lebesgue const?

  
  
\end{noframe}

\begin{noframe}\ft{Node choice and adaptivity}

Recap: poly approx.\ $f(x)$ on $[a,b]$: exist good \& bad node sets $\{x_j\}_{j=1}^N$

\sg

  \lb{Question}: Do {\em you} get to choose the set of nodes at which $f$ known?

\sg
  
  \gb data fitting applications: No \com{(or noisy variants: kriging, Gaussian processes, etc)}

  \qquad \com{use local poly (central region only!), or something stable (eg splines)} \hfill\whoc{\cite{GCbook}}
  
  \gb almost all else, interp., quadrature, PDE solvers: Yes
  \hfill \com{so pick good nodes!}

  \pause
  \vg
  
\lb{Adaptivity idea}
\hfill \com{global is \red{inefficient} if $f$ smooth in most places, but not everywhere}

\pause
\sg

 \only<3>{ \pig{3.2in}{figs/adap1} }
 \pause
 \pig{3.2in}{figs/adap2}
 \hfill
 \bmp{1.2in}
% Given tolerance $\eps$ (eg $10^{-12}$)

 automatically split
 
(recursively) panels

 until max err $\le\epsilon$

\sg
 
 \com{via test for local error}
 
 \emp

\vg
 
1D adaptive interpolator codes to try:

 \gb {\tt github:dbstein/function\_generator}
 \; \com{py+numba, fast} \hfill \who{Stein '19}

 \gb {\tt chebfun} for MATLAB
\; \com{big-$N$ Cheb.\ grids done via FFTs!}
\hfill  \who{Trefethen et al.}

\sg
 
App.:  replace nasty expensive $f(x)$ by cheap one! \hfill \com{optimal ``look-up table''}


\end{noframe}
    
% ---------------------------------------------------
\begin{frame}\ft{Global interpolation of periodic functions I}

  \com{Just did $f$ on intervals $[a,b]$. \;
    global interp.\ (\& integr., etc.) of smooth {\em periodic} $f$ differs!}

\hg

  Periodic: 
  $f(x+2\pi)=f(x)$ for all $x$, \quad $f(x) = \sum_{n\in\mathbb{Z}} \hat f_k e^{ikx}$
  \hfill \com{Fourier series}

\pause
  
  Instead of poly's, use \ora{truncated} series \;
  $\tilde f(x) = \sum_{|k|<N/2} c_k e^{ikx}$ \hfill\com{``trig.\ poly''}

  \sg

  \pause
\bmp{2in}
What's best you can do?

\quad get $N$ coeffs right $c_k=\hat f_k$

\sg

\quad error $\sim$ size of tail $\{\hat f_k\}_{|k|\ge N/2}$
\emp
\hfill
\pig{2.6in}{figs/fhat}

%\vspace{-1ex}

\pause

How read off $c_k$ from {\em samples} of $f$ on a grid?

\quad \com{uniform grid best (unlike for poly's!); non-uniform needs
  linear solve, slow $\bigO(N^3)$ effort}

%  How read off $\mbf{c} := \{c_k\}_{k=-N/2}^{N/2-1}$ from values $\y := \{f(x_j)\}_{j=1}^N$ at nodes?

  Uniform grid $x_j=\frac{2\pi j}{N}$, \; set
  $c_k  = \frac{1}{N}\sum_{j=1}^N e^{ikx_j} f(x_j)$ \hfill
  \com{simply $\mbf{c}=FFT[\mbf{f}]$}

\pause
  
\quad easy to show $c_k = \dots + \hat f_{k-N} + \hat f_k + \hat f_{k+N} + \hat f_{k+2N} + \dots$

\hfill $ = \; \hat f_k$ \com{ desired }
$+ \;\sum_{m\neq 0}\hat f_{k+mN}$ \;\com{ aliasing error, small if tail small}
%  due to discrete sampling}

\sg
\pause

Summary: given $N$ samples $f(x_j)$, interp.\ error = truncation + aliasing

\quad a crude bound is $\displaystyle\max_{x\in[0,2\pi)} |\tilde f(x)-f(x)| \le 2\sum_{|k|\ge N/2} |\hat f_k|$

\quad \com{ie error controlled by sum of tail}

\end{frame}


  % ---------------------------------------------------
\begin{noframe}\ft{Global interpolation of periodic functions II}

  As grow grid $N$, how accurate is it?
  \hfill 
  \com{just derived err $\sim$ sum of $|\hat f_k|$ in tail $|k|\ge N/2$}

\sg
  
  \quad \com{Now \;\; $\hat f_k = \frac{1}{2\pi}\int_{0}^{2\pi} f(x) e^{-ikx}dx =
  \frac{1}{2\pi}\int_{0}^{2\pi} f^{(p)}(x) \frac{e^{-ikx}}{(ik)^p}dx$
\hfill integr.\ by parts $p$ times}

\sg
  
  So for a periodic $f\in C^p$, \; \com{recall first $p$ derivs of $f$ bounded}
  
  %has $p$ bounded derivs
  \qquad $\hat f_k=\bigO(k^{-p})$, tail sum $\bigO(N^{1-p})$ %=\bigO(h^{p-1})$
  \;\;\com{$(p{-}1)$th order acc.}
  \hfill\who{better: \cite{tref}}
  % \;(could tweak)}

\vg
\pause

Example of: \hfill \framebox{$f$ smoother \;$\leftrightarrow$\; faster $\hat f_k$ tail decay \;$\leftrightarrow$\; faster convergence}

\pause

\sg

Even smoother case: $f$ analytic,\quad
\com{so $f(x)$ analytic in some complex strip $|\im x|\le \alpha$}

\quad then $\hat f_k = \bigO(e^{-\alpha|k|})$, exp.\ conv.\ $\bigO(e^{-\alpha N/2})$
\hfill \com{(fun proof: shift the contour)} 

\quad \com{as with Bernstein ellipse, to get exp.\ conv.\ {\em rate} need understand $f$ {\em off its real axis} (wild!)}

\sg
\pause

Smoothest case: ``band-limited'' $f$ with $\hat f_k=0$, $|k|>k_\tbox{max}$,

\quad then interpolant {\em exact} once $N>2k_\tbox{max}$

\vg
\pause

That's theory. In real life you always \ora{measure} your conv.\ order/rate!

\hg

Messages:

\gb $f$ smooth, periodic, global interpolation w/ uniform grid: spectral acc.\

\gb key to spectral methods.
FFT \com{cost $\bigO(N\log N)$} swaps from $f(x_j)$ \com{grid} to $\hat f_k$ \com{Fourier coeffs}

\end{noframe}



% ---------------------------------------------------
\begin{noframe}\ft{Flavor of interpolation in higher dims $d>1$}

  \bmp{2.2in}
  If you {\em can} choose the nodes:

\sg
  
  \quad tensor product of 1D grids

  \quad either global

  \quad or adaptively refined boxes %(quad-tree)
  
  \emp
  \pig{1in}{figs/2dugrid_lab}
  \quad
  \pig{0.9in}{figs/2dadaptgrid}
    
  \hspace{2.2in}
  \com{periodic, global}
  \hspace{.3in}
  \com{adaptive $p=6\times 6$ Cheby}

%\hfill\who{Babb et al '18}  but this is generic

\sg
\pause
If {\em cannot} choose the nodes:
    interp.\ $f(\x)$ from scattered data $\{\x_i\}$ is hard

    \hg

  \bmp{2.2in}
  \quad \com{eg google terrain: $f(\x)$ rough $\rightarrow$ garbage:}
% Bavarian pre-alps
%\hfill \com{see amusing jump in interp.\ type:}

\vg
\uncover<3->{%u

But if know $f$ smooth:

\quad \com{locally fit multivariate polynomials}

\sg

If also data noisy, many methods:

\quad  \com{kriging (Gauss.\ proc.), NUFFT, RBF\dots}

\sg

If also high dim $d\gg 1$:

%\quad\com{  sparse grids}

\quad\com{  tensor train, neural networks\dots}

\vg
}%u
  \emp
  \hfill
  \pig{2.2in}{figs/gterrain_lab}
  
 \end{noframe}

% ---------------------------------------------------
\begin{noframe}\ft{Numerical integration (back to $d=1$)}

  Task: eval.\ $\int_a^b f(x) dx$ accurately w/ least number of func.\ evals, $N$

\sg
  
``quadrature'': nodes $\{x_j\}$, weights $\{w_j\}$, s.t.
$\int_a^b f(x) dx \approx \sum_{j=1}^N w_j f(x_j)$

\sg
  
%Here, usually user gets to choose nodes $x_j$ \hfill\com{so pick best ones!}

Idea: \; get interpolant $\tilde f$ thru data $f(x_j)$ $\rightarrow$ {\em
    integrate that exactly}

\hfill  \com{``intepolatory quadrature''}

\pause
\vspace{-1ex}

\bmp{3.3in}
Examples:

\gb local piecewise linear $\rightarrow$ composite trapezoid rule

   \qquad \com{$w_j=h$ except $h/2$ at ends. \; low-order, err $\bigO(N^{-2})$, avoid!}
   \emp
   \hfill
   \pig{0.9in}{figs/trap}

\sg
   
   \gb $N$-node global poly $\rightarrow$ gives
% unique
   $\{w_j\}$ integrating degree $N{-}1$ exactly

   \qquad \com{$f$ analytic: err $\bigO(\rho^{-N})$ \hfill solve lin sys
     $V^T \mbf{w} = \{\int_a^b x^k dx\}_{k=0}^{N-1}$
     %\{monomial integrals\}
   } \; \who{Newton--Cotes}

\sg
   
   \gb better: ``Gaussian'' $\{x_j,w_j\}$ integrates deg.\ $2N{-}1$
   exactly! \hfill \com{err $\bigO(\rho^{-2N})$}

Adaptive quadrature (Gauss in each panel) excellent: \com{codes {\tt quadgk}, {\tt scipy}, etc}

\sg
\pause

\gb periodic case: $x_j=\frac{2\pi j}{N}$, $w_j=\frac{2\pi}{N}$ excellent
\quad \com{``periodic trap.\ rule''}

\qquad \com{easy to check integrates $e^{ikx}$ exactly for $|k|<N$, ``Gaussian''}

\qquad \com{$f$ analytic in $|\im x|<\alpha$ gives exp.\ conv.\ $\bigO(e^{-\alpha N})$, twice as good as interp!}

\pause

\quad \lb{demo}: \; {\tt \small N=14; sum(exp(cos(2*pi*(1:N)/N)))/N - besseli(0,1)}

\hspace{1.1in} {\tt \small ans = 1.3e-15}

  
\end{noframe}


% ---------------------------------------------------
\begin{frame}\ft{Advanced integration}

  \gb custom quadr.\ for singularity \; \com{eg $f(x) = \mbox{smooth}\cdot|x|^{-1/2}$}
\hfill \who{Rokhlin school}
  
  \qquad or for arb.\ set of funcs.\ \com{``generalized Gaussian quad.''}
\hfill \who{CCM: Manas Rachh}

\gb high-order end-corrections to uniform trap.\ rule \hfill \who{Alpert '99}

\gb oscillatory functions: deform contour to $\mathbb{C}$\; \com{``numerical steepest descent''}

%\gb changes of variables \com{eg tanh-sinh transform to remove endpoint singularities}

\dots

\sg
\pause

\lb{Higher dimensions $d>1$}
\hfill \com{code: {\tt integral2}, etc, {\tt quadpy}}

For $d \lesssim 5$, tensor product quadr.\ of 1D $n$-node grids in each dim

\quad \com{other coord systems: eg sphere can use tensor product in $(\theta,\phi)$. \; Or: iterate over dims.}

adaptivity works: automatically refine boxes
 \; \com{but soon enter research territory!}

$\int_\Omega f(\x)d\x$ in nasty domain $\Omega\subset \mathbb{R}^d$ ?
\; \com{FEM meshing, blended conforming grids\dots}
%  boxes blending to boundary-conforming grid}

    
\vg
\pause

\bmp{3.9in}
  \lb{Much higher $d\gg 1$}

  tensor prod: exp.\ \# $f$ evals.\ $N=n^d$ kills you :( \;\com{``curse of dim.''}

  
  \gb  ``sparse grids'' \; \com{scale better as $N \sim n (\log n)^d$} \hfill \who{Smolyak '63}
% scaling according to Gerstner review article.
  \emp
  \hfill\pig{.6in}{figs/sparsegrid2d}
  
  %\com{rely on funcs aligning w/ axes}
  
  \gb (quasi-)Monte Carlo: $\sum_{j=1}^N f(\x_j)$, for random $\x_j$
  \;\com{err $\bigO(N^{-1/2})$, slow conv!}

  \qquad \com{importance sampling \who{Thurs am session}, custom transformations\ldots}
  
% how does variance prefactor scale w/ dim d?
  
 % \quad \com{Note MCMC does not easily give $\int f(\x) d\x$, just ratios of such}

  
\end{frame}

% ---------------------------------------------------
\begin{noframe}\ft{Numerical differentiation}

  Task: given ability to eval.\ $f(\x)$ anywhere, how get $\nabla f(\x)$ ?
  \hfill\com{assume smooth}

%\qquad \com{obviously again we have to assume $f$ smooth!}
\sg
\pause

\lb{Finite differencing} idea, 1D:\;\; $f'(x) = \frac{f(x+h)-f(x-h)}{2h} + \bigO(h^2)$
\hfill\com{Taylor's thm}


\bmp{1.8in}
\com{``centered difference'' formula}

\vg

Want smallest error:

suggests taking $h\to 0$ ?

\vg

Let's see how that goes\ldots
\vspace{8ex}

\emp
\pause
\hfill\pig{2.7in}{figs/fderr}

\sg

\gb shrinking $\bigO(h^2)$ error gets swamped by a new growing error\dots what?

\pause

\gb CPU arithmetic done only to relative ``rounding error'' $\epsilon_\tbox{mach}\sim 10^{-16}$

\gb subtracting v. close $f(x+h)$ and $f(x-h)$: ``catastrophic cancellation''

\gb balance two error types: $h_\tbox{best} \sim \epsilon_\tbox{mach}^{1/3} \sim 10^{-5}$

\sg

\com{Essential reading: floating point, backward stability}\; \whoc{\cite[Ch.~5--6]{GCbook} \cite[Ch.~12--15]{nla}}

\end{noframe}


% ---------------------------------------------------
\begin{noframe}\ft{High-order (better!) differentiation, $d=1$}

As w/ integration: get interpolant $\rightarrow$ differentiate it exactly
\hfill \whoc{\cite[Ch.~6]{tref}}

\quad \com{Get $N\times N$ matrix $D$ acting on func.\ values $\{f(x_j)\}$ to give $\{f'(x_j)\}$. Has simple formula}

\sg
\pause

\bmp{1.4in}
Examples:

\sg

\quad \com{$N$ Chebychev nodes}

\quad \com{in $[-1,1]$}

\sg

shown: max error in $f'$

\vspace{20ex}

\emp
\hfill
\pig{3in}{figs/trefchebdiff_lab.png}


\gb for $N$ large, the dense $D$ is never formed, merely applied via FFT

\qquad \com{spectral solvers for ODE/PDEs. codes: {\tt chebfun}, {\tt PseudoPack}, {\tt dedalus}... Lecture II}

\end{noframe}


% ---------------------------------------------------
\begin{frame}\ft{Summary: we scratched the surface}

\vg
  
Can integrate \& differentiate smooth funcs given only point values $f(x_j)$

\sg

Both follow from building a good (fast-converging) interpolant

\sg

For $f$ smooth in 1D, can \& should easily get many (10+) digits accuracy

\vspace{3ex}

\lb{Concepts:}

\sg

convergence order/rate \quad  \com{how much effort will you have to spend
  to get more digits?}

smoothness \hfill \com{smooth $\Leftrightarrow$ fast convergence; \; nonsmooth needs custom methods}

  global \hspace{1in} \com{(one interpolation formula/basis for the whole domain)}

  \quad vs local \hspace{.8in} \com{(distinct formulae for $x$ in different regions)}

  spectral method \hspace{.3in} \com{global, converge v.\ fast, even non-per.\ can exploit FFT}
%(even non-periodic Cheb.\ case),
    %er than any fixed order}
%    $N^{-p}$ for any $p$}
  
  adaptivity \hfill \com{auto split boxes to put nodes only where they need to be}

  rounding error \& catastrophic cancellation \quad \com{how not shoot self in the foot}

  tensor products for 2D, 3D \quad \com{for higher dims: randomized/NN/TN}\;\who{Th/Fr sessions}
  
%for $d\gg 1$ high dims, see other sessions: random \com{Monte Carlo} and neural networks.

  \vspace{3ex}

See recommended books at end, and come discuss stuff!

\end{frame}





% ---------------------------------------------------
\begin{frame}\ft{LECTURE II: numerical differential equations}

\lb{Motivation}

Produce numerical approximations to the solutions of ODEs/PDEs.

\vspace{1em}

\lb{Goals for today}

Basic overview of how different methods work.

Understand error properties and suitability for different equations.

\pause

\vspace{1em}

\lb{Families of methods}:

\begin{itemize}
	\item Finite Difference Methods \com{For time \& space.}
	\pause
	\item Finite Element Methods \com{Very general}
	\pause
	\begin{itemize}
		\item Finite Volume Methods \com{Fluids}
		\pause
		\item ``Traditional'' Finite Elements \com{Mechanics}
		\pause
		\item ``Modern'' Finite Elements \com{Higher order}
		\pause
		\item Spectral Methods \com{Best accuracy for smooth solutions}
	\end{itemize}
	\pause
	\item Boundary Integral Methods \com{Linear problems w/ boundary data}
\end{itemize}

\end{frame}

% ---------------------------------------------------
\begin{noframe}\ft{Reminder of types and applications of diff.\ eq.}

% feel free to scratch/edit, but some overview is needed
  
\gb  ODEs: \com{eg pendulum} $u''(t) + \sin(u(t)) = 0$
	
	\qquad Task: solve $u(t)$ given initial conditions \quad \com{e.g. $u(0)=1$, $u'(0)=0$}
	
	\pause
	
	\qquad \com{Others: local chemical/nuclear reactions ($\mbf{u}(t)$ is vector of multiple components)}
	
\pause

\sg

\gb Time-independent PDEs: \com{eg Poisson eqn} \; $\Delta u (\mbf{x}) = g(\mbf{x})$

	\qquad Task: solve $u(\mbf{x})$ given forcing, boundary conditions

	\qquad \com{Steady state of heat/diffusion, Gauss's law for conservative forces}

	\qquad \com{$u(\mbf{x})$ is chemical concentration, gravitational/electric potential}
	
	\qquad \com{$\Delta u$ means Laplacian $\partial^2 u/\partial x^2 + \partial^2 u/\partial y^2 + \dots = $ curvature of $u$}
	
  	\qquad \com{$g(\x) = $ volume source of chemical, mass or charge density}
  	
  	\pause

	\qquad \com{Others: Stokes eqn for velocity field $\mbf{u}$ in viscous fluid}
  
	\qquad \com{Others: t-indep.\ Schr\"odinger eqn for quantum systems: $\Delta \psi = (V -E) \psi$}

\sg

\pause

\gb Time-dependent PDEs: \com{eg advection-diffusion} \; $\partial_t c + \nabla \cdot (\mbf{u} c) = \Delta c$

	\qquad Task: solve $c(\x,t)$ given initial \& boundary conditions
	
	\qquad \com{Others: Navier-Stokes, magnetohydrodynamics, ...}

\sg

\pause

[Mike will overview the three flavors of PDE in next talk]

\sg

BCs: simple (eg periodic cube) vs complicated ($u=0$ on a nasty surface)

  
\end{noframe}

% ---------------------------------------------------
\begin{frame}\ft{Typical solution strategies}

Time-independent PDEs:
\begin{enumerate}
	\item Discretize variables (grid points, cells, basis functions)
	\item Discretize operators/equations (derivatives)
	\item Solve resulting algebraic system
\end{enumerate}

\pause

Time-dependent PDEs: \com{``method of lines''}
\begin{enumerate}
	\item Discretize variables (grid points, cells, basis functions)
	\item Discretize operators/equations (derivatives)
	\item Solve resulting coupled ODEs
\end{enumerate}

\pause

ODEs:
\begin{itemize}
	\item Treat spatial problems as time-indep.\ PDEs \com{``boundary value problems''}
	\pause
	\item Evolve temporal problems with finite differences \com{``initial value problems''}
\end{itemize}

\end{frame}

% ---------------------------------------------------
\begin{frame}\ft{Finite difference methods}

Basic viewpoint:

\gb Discretize variables on a discrete grid

\gb Construct Taylor-series approximations to values at neighboring points

\gb For N points, expand to N terms (error $\bigO(h^N)$)

\gb Eliminate to get approximation to $d$-th derivative (error $\bigO(h^{N-d})$)

\pause 

\sg

{\small

E.g. Centered differences on 3 points: $x-h$, $x$, $x+h$
%
\begin{align}
	u(x + h) &= u(x) + u'(x) h + u''(x) h^2 / 2 + \bigO(h^3) \\
	u(x - h) &= u(x) - u'(x) h + u''(x) h^2 / 2 + \bigO(h^3)
\end{align}

\pause

To approximate $u'(x)$, subtract to eliminate $u''(x)$:
%
\begin{equation}
 u'(x)= \frac{u(x+h) - u(x-h)}{2 h} + \bigO(h^2)
\end{equation}

\pause
 
To approximate $u''(x)$, add to eliminate $u'(x)$:
%
\begin{equation}
	u''(x) = \frac{u(x+h) - 2 u(x) + u(x-h)}{h^2} + \bigO(h^2)
\end{equation}

}

\begin{center}
	\com{Extra order here due to symmetry}
\end{center}

\end{frame}

% ---------------------------------------------------
\begin{frame}\ft{Finite difference methods}

Alternate viewpoint:

\gb Discretize variables on a discrete grid

\gb Construct interpolating polynomial of N nearest points. 

\qquad \com{Unique, degree N-1.}

\gb Differentiate the local interpolant to approximate derivatives.

\pause

\vspace{2em}

{\small

E.g. Centered differences using 3 points:

\begin{center}
\pig{4in}{figs/fd_local_interp}
\end{center}


}


\end{frame}

% ---------------------------------------------------
\begin{frame}\ft{Implicit \& Explicit Timestepping}

Consider temporal ODE $u'(t) = f(u(t))$.

Timesteppers solve using finite differences to advance $u_n \rightarrow u_{n+1}$

\sg

\pause

Explicit schemes: just need $f(u_n)$. \com{Simple but unstable for large steps}

{\small
\gb E.g.\ forward Euler: use 1st-order forward difference
%
\begin{align}
	u'(t) &= - \lambda u(t) \quad \lambda > 0 \\
	u_{n+1} - u_n &= - k \lambda u_n \\
	u_{n+1} &= (1 - k \lambda) u_n
\end{align}
}
\begin{center}
	\com{$k \lambda < 2$ for stability}	
\end{center}

\sg

\pause 

Implicit schemes: require solving $f(u^{n+1}) = ...$ \com{Stable but expensive}

{\small
\gb E.g.\ backward Euler: use 1st-order backward difference
%
\begin{align}
u'(t) &= - \lambda u(t) \quad \lambda > 0 \\
u_{n+1} - u_n &= - k \lambda u_{n+1} \\
u_{n+1} &= (1 + k \lambda)^{-1} u_n
\end{align}
}

\end{frame}

% ---------------------------------------------------
\begin{frame}\ft{Finite difference methods}

\vspace{2em}

\bmp{3.5in}
\gb Extends to multiple dimensions with regular grids

\gb Simple to adjust order of accuracy

\gb Some more advanced techniques:

\qquad \gb Conservative schemes

\qquad \gb Select stencils term by term \com{``upwinding''}

\qquad \gb Adaptive stencil selection for jumps \com{``WENO''}

\gb Restricted to simple geometries / well-structured grids

\emp
\hfill
\pig{1in}{figs/fd}  

\vspace{2em}

\pause

Resources: LeVeque ``Finite Difference Methods for ODE/PDE''

Codes: e.g.\ Pencil code (magnetohydrodynamics)

\end{frame}

% ---------------------------------------------------
\begin{frame}\ft{Finite element methods}

\bmp{3.5in}

\gb Partition domain into elements. \com{Unstructured}

\gb Represent variables with basis functions on elements:
%
\begin{equation}
	u(\mbf{x}) = \sum_{n=1}^{N} u_n \phi_n(\mbf{x})
\end{equation} 

\qquad \com{``Trial functions'' $\phi_n$ usually polynomials on each element}

\emp
\hfill
\pig{1in}{figs/fem}

\pause

\sg

\gb Solve equations using Galerkin/weighted-residual method:
%
\begin{equation}
	\partial_t u(\mbf{x}) + L u(\mbf{x}) = f(\mbf{x})
\end{equation}
%
\begin{equation}
	\int \psi_m (\mbf{x}) \left[\partial_t u(\mbf{x}) + L u(\mbf{x}) - f(\mbf{x})\right] \,d\mbf{x} = 0
\end{equation}
%
\qquad \com{For all ``test functions'' $\psi_m$}

\pause

\gb Solve resulting algebraic system:
%
\begin{equation}
	M \cdot \partial_t \mbf{u} + S \cdot \mbf{u} = M \cdot \mbf{f}
\end{equation}
%
\qquad \com{``Mass matrix'' $M$, ``stiffness matrix'' $S$}

\end{frame}

% ---------------------------------------------------
\begin{frame}\ft{Finite volume methods}

\gb Piecewise constants inside elements as test/trial functions

\qquad \com{$M = I$, easy explicit formulation}

\pause

\gb Integrate flux terms by parts:
%
\begin{equation}
	\int \psi_m \nabla \cdot \mbf{j} \,d\mbf{x} = \int_{\Omega_i} \nabla \cdot \mbf{j} \,d\mbf{x} = \int_{\delta \Omega_i} \mbf{n} \cdot \mbf{j} \,dS
\end{equation}

\gb Requires integrating fluxes at cell interfaces (usually 2nd order)

\qquad \com{Many methods for Riemann solvers/flux reconstruction: TVD, ENO, WENO, ...}

\pause

\gb Exactly conservative: good for hyperbolic PDEs.  Widely used in CFD.

\gb Similar the finite difference on structured meshes.

\gb Hard to build high-order schemes on unstructured meshes.

\sg

\pause

Codes: Arepo, Athena, OpenFOAM

\com{Many local experts in CCA!}

\end{frame}

% ---------------------------------------------------
\begin{frame}\ft{Finite element methods}

Traditional FEM

\gb Use piecewise linear ``tent'' functions. \com{Continuous, 2nd order}

\gb ``Weak form'' from integrating by parts:
%
\begin{equation}
	\int \psi_m \nabla^2 u \,d\mbf{x} = - \int \nabla \psi_m \cdot \nabla u \,d\mbf{x}
\end{equation}

\qquad \com{Lowers order of derivatives, allows piecewise linear representation}

\gb Not conservative and $M \neq I$, need implicit schemes or to invert $M$

\gb Easy to adjust order of accuracy. \com{Use higher degree polynomials, ``p adaptivity''}

\vspace{1em}

Modern research: high-order FEM

\gb Discontinuous Galerkin (FVM + FEM): high order inside elements, but allow discontinuities. \com{Need Riemann solvers again}

\gb Spectral elements: very high order internal representations

\vspace{1em}

Codes: FEniCS, deal.II

\end{frame}

% ---------------------------------------------------
\begin{frame}\ft{Spectral methods}

\small
  
\gb Expand variables in global basis functions (FEM with one element)

\gb Solve Galerkin projection of equations.  \com{But don't integrate by parts}

\gb \textbf{Exponential} accuracy for smooth solutions

\pause

\vspace{1em}

Periodic intervals: Fourier series for test/trial functions. \com{Fast w/ FFT}

\qquad \com{$M$ and $S$ matrices typically diagonal, even in multiple dimensions!}
%
\begin{equation}
	\nabla^2 \exp(i \mbf{k} \cdot \mbf{x}) = - k^2 \exp(i \mbf{k} \cdot \mbf{x})
\end{equation}

\pause

\vspace{0.5em}

Non-periodic intervals: Chebyshev polynomials $T_n(x)$. \com{Fast w/ DCT}

\qquad \com{Traditional: ``collocation'' using values at Chebyshev nodes. Dense matrices.}

\qquad \com{Modern: $M$ and $S$ banded with right choice of test functions}.

\begin{center}
	\pig{2in}{figs/fig_chebyshev_deriv_matrices}
\end{center}

Other geometries: other polynomials, spherical harmonics, ...


\end{frame}

% ---------------------------------------------------
\begin{frame}\ft{Spectral methods}

\small

\gb \textbf{Exponential} accuracy for smooth solutions. \com{Need to regularize discontinuities}

\begin{center}
	\pig{1.3in}{figs/fig_OT_image}
	\hspace{1em}
	\pig{2in}{figs/fig_OT_line}
\end{center}

\pause

\gb Restricted to simple geometries. \com{Boxes, spheres, disks, ...}

\gb Very flexible in terms of equations.

\gb Not exactly conservative... but very accurate. \com{Use conservation as a diagnostic!}

\pause

\vspace{1em}

Modern research: sparse methods for arbitrary equations in more geometries.

\pause

\sg

Resources: Boyd ``Chebyshev and Fourier Spectral Methods''

Codes: Chebfun (MATLAB), ApproxFun (julia), Dedalus (Python)

\end{frame}

% ---------------------------------------------------
\begin{frame}\ft{Boundary integral methods}

Use knowledge of PDEs in constructing solutions:

\gb Linear PDEs dominated by boundary terms

\gb Solutions involve integrals of fundamental solution (Green's function):

\qquad \com{Reduced dimensionality. Improved conditioning. Low-rank iterations and fast methods.}

\sg
\sg

\bmp{3in}
\small
E.g. for Poisson's equation: $\Delta u(\mbf{x}) = f(\mbf{x})$
%
\begin{equation*}
	u(\mbf{x}) = \int G(\mbf{x}, \mbf{y}) f(\mbf{y}) \,d\mbf{y}
\end{equation*}
%
\begin{equation*}
	\Delta G(\mbf{x}, \mbf{y}) = \delta(\mbf{x} - \mbf{y}), \quad G(\mbf{x}, \mbf{y}) = \frac{1}{4 \pi |\mbf{x} - \mbf{y}|}
\end{equation*}

\emp
\hfill\pig{1.5in}{figs/bie}

\sg
\sg

Examples: Stokes flow, Helmholtz equation, Maxwell equations 

\qquad \com{Usually homogeneous media}

\sg

Many experts in CCM \& CCB.  See Jun Wang's talk later today!

\end{frame}

% ---------------------------------------------------
\begin{frame}\ft{Summary}



  
\end{frame}

% ---------------------------------------------------
\begin{frame}\ft{Recommended accessible reading}

\bibliographystyle{amsalpha}  % abbrv
\bibliography{numpde}

\vspace{4ex}

This document: \; \url{https://github.com/ahbarnett/fwam-numpde}

\sg

See {\tt code} directory for MATLAB code used to generate figures


\end{frame}



\end{document}
